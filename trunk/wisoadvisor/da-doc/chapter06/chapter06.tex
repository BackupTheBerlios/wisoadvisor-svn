\section{Performance Optimizer}
\label{sec:PerformanceOptimizer}

\subsection{Konzeption und Prototyp}
\subsubsection{Konzeption}
Das vorliegende Konzept zum "`Performance Optimizer"' (POPT) \abbrev{POPT}{Performance Optimizer} entstammt ebenfalls der bereits zuvor erw�hnten Diplomarbeit \cite{Berchtold:2007}.

Dem Konzept liegt die betriebswirtschaftliche Theorie zum Controlling-Instrument der "`Balanced Scorecard"' (BSC)\abbrev{BSC}{Balanced Scorecard} und deren Implementierung in SAP Strategic Enterprise Management zu Grunde \cite<vgl. hierzu>[S. 13-36]{MertensMeier:2005}. 

Abbildung \ref{fig:sap_bsc_mertensmeier2005} zeigt exemplarisch eine Balanced Scorecard, auf der die vier traditionellen Perspektiven (hier z. B. "`Financial Perspective"') und die zugeh�rigen Strategieelelemente (z. B. "`Become no. one supplier in installed base"') dargestellt werden. Unterhalb des Strategieelemente befinden sich schlie�lich die Kennzahlen (z. B. "`Market Rank Software"'). 

\begin{figure}[h]
	\centering
		\includegraphics[width=0.66\textwidth]{chapter06/sap_bsc_mertensmeier2005.png}
	\caption{Balanced Scorecard in SAP Strategic Enterprise Management}
	\label{fig:sap_bsc_mertensmeier2005}
	\cite{MertensMeier:2005}
\end{figure}

Das Auff�lligste an dieser Implementierung von SAP ist, dass die einzelnen Aggregationsebenen durch Icons farblich gekennzeichnet sind (das Icon einer �bergeordneten Ebene entspricht dabei i. d. R. \abbrev{i. d. R. }{in der Regel} dem des schlechtesten ihr untergeordneten Wertes). Neben dieser �bersicht bietet die SAP-Implementierung auch eine sog. Analysesicht, auf der s�mtliche Perspektiven, Strategielemente und Kennzahlen in einer Baumstruktur dargestellt werden \cite<siehe>[S. 18]{MertensMeier:2005}.

Eben diese multiperspektivische, auf mehreren Aggregationsebenen angesiedelte und optisch ansprechende Darstellung von Leistungswerten bildet die Grundlage des Konzepts zum "`Performance Optimizer"': Dem Studenten sollen in einer BSC, deren einzelne Perspektiven durch die Studienbereiche seines Studiengans (vgl. \ref{sec:StudienverlaufAnDerWiSo}) repr�sentiert werden, Gesamt-, Teil- und Einzelnoten visualisiert werden.

Als weitere Bestandteile des POPT wurden festgelegt \cite<vgl.>[S. 101-105]{Berchtold:2007}:
\begin{itemize}
  \item \textbf{Simulation:} Zus�tzlich zur Darstellung der tats�chlichen Noten soll der Student die M�glichkeit bekommen, die seiner Meinung idealen und schlimmsten Ergebnisse seiner Pr�fungen anzugeben, um so deren Auswirkungen auf sein Notengef�ge sehen zu k�nnen.
  \item \textbf{Benchmarking:} Mit einer Benchmarking-Komponente w�re der Student in der Lage, seine eigene Position in der gesamten Notenstruktur einer Pr�fung auszumachen. Derartige Informationen wurden bisher nur von einzelnen Lehrst�hlen der WiSo bereitgestellt; die Ver�ffentlichung von Pr�fungsergebnissen -- sei es als Aushang am Schwarzen Brett oder als Download -- wurde jedoch im Sommersemester 2007 durch das zust�ndige Pr�fungsamt unterbunden. Eine Benchmarking-Komponente im Rahmen des POPT w�re also nicht nur ein ad�quater Ersatz, sondern auch einheitlich f�r alle Pr�fungen verf�gbar.
  \item \textbf{Identifikation von Strukturbr�chen im Leistungsprofil:} Der Grundgedanke hinter diesem Aspekt ist, nicht nur ex post "`Ausrutscher"' im Notengef�ge zu identifizieren, sondern aus solchen vereinzelten Einbr�chen Tendenzen und Warnungen f�r k�nftige Pr�fungen abzuleiten.
  \item \textbf{Abgabe von Feedback:} Insb. \abbrev{insb.}{insbesondere} im Zusammenhang mit den zuvor genannten Strukturbr�chen w�re es notwendig, vom Studenten ein transferf�rderndes Feedback abgeben zu lassen, mit dem er eine schlechte Pr�fungsleistung begr�nden muss.
  \item \textbf{Personalisierung durch individuelles Setup:} Dieser Aspekt betrifft prim�r die Visualisierung der Gesamt- und Teilergebnisse; der einzelne Benutzer soll selbst entscheiden k�nnen, wie ihm seine Noten im Einzelnen dargestellt werden. 
\end{itemize}

\subsubsection{Prototypische Implementierung}

\subsubsection{Weitere Anforderungen an das Produktivsystem}

Generell problematisch an der Sichtweise des "`Performance Optimizer"' als Balanced Scorecard ist in Bezug auf die studentische Situation, dass die BSC eine Ausgewogenheit impliziert, d. h. eine Gleichberechtigung ihrer Dimensionen. Aus der Verteilung der ECTS-Punkte auf die einzelnen Studienbereiche der Bachelor-Studieng�nge wird jedoch ersichtlich, dass sich diese dramatisch in ihrer Gewichtung unterscheiden. Das Adjektiv "`balanced"' ist somit nicht angebracht. Letztlich handelt es sich beim POPT prim�r "`nur"' um eine um Durchschnittswerte und H�ufigkeitsverteilungen erg�nzte \textbf{Noten�bersicht} \cite<zur Problematik mangelnder sprachlicher Pr�zision siehe>[S. 34-39]{Mertens:2005}.

Problematisch an der Simulationskomponente scheint, dass dem Studenten zus�tzlich zum bereits im Pr�fungsplan abgebenen Erwartungswert (vgl. \ref{sec:PlanerKonzeption}) jetzt ... ***


\subsection{Umsetzung des Konzepts im WiSo@visor}

\subsection{Ausblick und Erweiterungsm�glichkeiten}
