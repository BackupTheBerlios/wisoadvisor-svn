\section{Studieneingangsberatung f�r Erstsemester}
\label{sec:Studieneingangsberatung}

\subsection{Konzept und Prototyp}
\subsubsection{Konzeption}
Das vorliegende Konzept zur Studieneingangsberatung (SEB) \abbrev{SEB}{Studieneingangsberatung} entstammt einer Studienarbeit �ber innovative Ans�tze der Informationsvermittlung \cite{Nursinski:2007}.

Der Autor beruft sich bei seiner Konzeption prim�r auf ...

\subsubsection{Prototypische Implem}
Der Prototyp f�r die SEB besteht aus statischen HTML-Seiten, \abbrev{HTML}{Hypertext Markup Language} die mit verweissensitiven Grafiken und teilweise mit JavaScript-Elementen interaktiv gestaltet werden. Besonderes Augenmerk wird auf die Navigation gerichtet, sodass der Benutzer immer genau wei�, wo er sich in der Navigation befindet. 

Layout und Design des Prototypen sind an den Internetauftritt des Lehrstuhls Wirtschaftsinformatik III angelehnt. 

\subsection{Weitere Anforderungen an das Produktivsystem}

Da es sich bei der SEB im Gegensatz zu den sp�ter erl�uterten Konzepten ausschlie�lich um ein Informationsportal handelt, steht bei der Implementierung nicht die Frage nach der Anwendungslogik im Vordergrund, sondern zum einen -- wie in Konzept und Prototyp dargelegt -- die der Darstellung der relevanten Inhalte, und zum anderen die der Wartbarkeit und Pflege dieser Informationen.

\subsection{Umsetzung des Konzepts im WiSo@visor}

\subsubsection{Realisierung als Wiki}
Bei der Implementierung der SEB stand die Frage nach der technischen Gestaltung im Vordergrund: Die Inhalte sollten schnell und unb�rokratisch ge�ndert werden k�nnen. F�r den bestehenden WiSo@visor existiert derzeit allerdings noch keine Administrationsumgebung, in der die Textbausteine gepflegt werden k�nnen; �nderungen m�ssen hier direkt in der Datenbank erfolgen.

% a bisserl �ber verschiedene Wikis schwallen; auf Vergleichsseite verweisen
Im Konzept wurde bereits ber�cksichtigt, dass die Inhalte der SEB...

\subsubsection{Modifikationen an TigerWiki}
\label{wiki-mod}

Obwohl TigerWiki bereits alle grundlegenden Formatierungsm�glichkeiten mitbrachte, waren dennoch einige Modifikationen an der Software notwendig:
\begin{itemize}
\item Damit die im gesamten Wiso@visor verwendeten Grafiken nicht an verschiedenen Orten im Verzeichnisbaum liegen, lag es nahe, auch die im Wiki zu verwendenen Grafiken im Grafikverzeichnis des Advisor abzulegen. Dies aber hatte den Nachteil, dass im Code einer Wiki-Seite nun immer der relative Pfad zur Grafik mitangegeben werden musste. In der Konfigurationsdatei des Wikis ist nun ein \textbf{Pfad f�r Grafiken} definiert, der allen Grafiken und sonstigen Ressourcen vorangestellt wird. Das Warten der Seiteninhalte wird so erheblich erleichtert.
\item TigerWiki kann Inhalte nur untereinander darstellen. Nun unterst�tzt das Wiki einfache (einzeilige) \textbf{Tabellen}, um Inhalte auch nebeneinander darstellen zu k�nnen. Verwendet wird dies beispielsweise bei den Aussagen der Professoren, wo neben dem jeweiligen Statement auch ein Foto platziert wird.
\item Die verweissensitiven Grafiken des Prototypen erfordern aufw�ndigen HTML-Code. F�gt man diesen Code �ber das HTML-Element von TigerWiki ein, wird die Bearbeitung einer Seite sehr schnell un�bersichtlich. Der komplette Code dieser Grafiken wurde nun in externe HTML-Dateien ausgelagert; diese kann �ber eine \textbf{Include-Anweisung} eingef�gt werden. Die Konsequenz daraus ist, dass die bearbeitbare Wiki-Seite weitgehend frei von HTML-Code bleibt.
\end{itemize}

Diese Modifikationen wurden dem urspr�nglichen Entwickler der Software zugesandt und werden m�glicherweise in eine k�nftige Version des Systems integriert.

\subsubsection{Kurzreferenz zu TigerWiki}

Wie bereits erw�hnt, war es vor allem seine Einfachheit, die TigerWiki als Basis f�r das Wiki zur SEB so attraktiv machte. Tabelle \ref{tab:SyntaxDerWikiEintr�ge} gibt einen �berblick �ber die Kommandos, die nun zum Editieren der Seiten bereitstehen \cite<vgl.>{tigerwiki:aide}.


\begin{table}[ht]
\small
	\centering
		\begin{tabular}{|l|l|l|}

			\hline
			\textbf{Element}        & \textbf{Syntax}													& \textbf{Beispiel} \\
			\hline
			�berschrift             & ! (gro�)  															& !�berschrift 1 \\
			                        & !! (mittel) 														& !!�berschrift 1.1 \\
			                        & !!! (klein)														  & !!!�berschrift 1.1.1 \\			                        
			\hline
			Format                  & \lq\lq~(kursiv)  										    & \lq\lq kursiv \\
			                        & \lq\lq\lq~(fett)  										  & \lq\lq\lq fett \\
			\hline
			Liste                   & * (nicht nummeriert)  									& *erstes Element \\
			                        & \# (nummeriert)  											  & *zweites Element \\
			\hline
			Grafik                  & [Dateiname] (intern)										& [profamberg.jpg] \\
			                        & [URL] (extern)													& [http://www.foo.bar/profamberg.jpg] \\
			\hline
			Link                    & [Seite] (intern) 												& [Lehrveranstaltungen] \\
															& [Beschreibung|?page=Seite] (intern) 		& [Planspiel|?page=Planspiel] \\
															& [URL] (extern) 													& [http://www.wiso.uni-erlangen.de] \\
															& [Beschreibung|URL] (extern)							& [WiSo|http://www.wiso.uni-erlangen.de] \\
			\hline
			Tabelle                 & [TABLE:Spalte1||Spalte2]								& [TABLE:[profamberg.jpg]||Prof. Amberg] \\

			\hline
			Textdatei               & [INCLUDE:Dateiname]	  							    & [INCLUDE:planspiel.map] \\

			\hline
			HTML-Code               & \{...\}	  							    					  & \{<br><br>\} \\

			\hline					
		\end{tabular}
	\caption{Syntax der Wiki-Eintr�ge}
	\label{tab:SyntaxDerWikiEintr�ge}
\end{table}

Wie in \ref{wiki-mod} erw�hnt, ist zu beachten, dass das Wiki interne Ressourcen, also Grafiken und Textdateien, im voreingestellten Grafikverzeichnis sucht. So wird z. B. die Eingabe \texttt{[profamberg.jpg]} ersetzt durch \texttt{<img src="'../grafik/wiki/profamberg.jpg"'/>}.
