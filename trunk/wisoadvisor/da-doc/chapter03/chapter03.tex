\section{Grundlegendes zum WiSo@visor}
\label{sec:GrundlegendesAdvisor}

\subsection{Entstehung und Evolution}

\subsection{Einordnung in das Verst�ndnis dieser Arbeit}
Wie in \ref{sec:ControllingKonzept} erl�utert, besteht die erste S�ule des Controlling-Konzeptes aus der Studieneignungsberatung. Der angehende Student einerseits inhaltlichen Einblicke in das Studium, andererseits ermittelt ihm das System auf der Grundlage seiner Antworten Tendenzen, f�r welche Studienrichtungen er m�glicherweise besonders geeignet w�re.

[...]

\subsection{Technologische Aspekte}

\subsubsection{Entwicklungsumgebung}

Die Ausf�hrungen der beiden folgenden Kapitel st�tzen sich auf die technische Dokumentation zum WiSo@visor \cite[S. 52-77]{Strecker:2006}.

Ein genereller Anspruch an die Software war, webbasierte Eignungsberatung auf Basis von einfacher HTML-Technologie zu betreiben. STRECKER stellt in einem Vergleich existierender �hnlich gelagerter Beratungstests fest, dass diese fast ausschlie�lich das weit komplexere Flash einsetzen, was sich h�ufig negativ auf die Bedienbarkeit auswirkt \cite[S. 43-45]{Strecker:2006}. 

So kann man den WiSo@visor als eine beinahe "`klassische"' LAMP-Applikation bezeichnen (LAMP f�r die Kombination Linux-Apache-mySQL-PHP): 

Der WiSo@visor wurde mit der Skriptsprache \textbf{PHP}\abbrev{PHP}{PHP: Hypertext Processor} in der Version 5 entwickelt. Diese Sprache hat nicht nur den Vorteil, kostenlos zu sein, sondern ist vor allem plattformunabh�ngig einsetzbar; es existieren Plugins f�r alle g�ngigen Webserver. Bereits in sehr fr�hen Versionen war es mit PHP ohne gro�en Aufwand m�glich, auf verschiedene Datenbanksysteme zuzugreifen. Die j�ngeren Versionen von PHP forcieren vor allem objektorientierte Ans�tze.

Sowohl in der Entwicklungs- als auch in derzeitigen Produktivumgebung wird der \textbf{Apache-Webserver} eingesetzt, der mit �ber 50\% Marktanteil nach wie vor der am h�ufigsten eingesetzte Webserver ist \cite{netcraft:2007}.

Als Datenbanksystem fungiert derzeit \textbf{mySQL}; die Datenbankschicht ist jedoch so gestaltet, dass sie ohne gro�en Aufwand auf die Open-Source-Datenbank Firebird konvertiert werden kann. Fortschrittlichere Konzepte, wie etwa ACID-konforme Transaktionen, werden vom WiSo@visor -- einschlie�lich seines Frameworks (siehe hierzu \ref{sec:AdvisorArchitektur}) -- nicht genutzt.

Zum Erstellen von PDF-Dokumenten\abbrev{PDF}{Portable Document Format} und zum Versenden von E-Mails werden in die Software zus�tzlich \textbf{Bibliotheken von Drittanbietern} einbezogen.

\subsubsection{Systemarchitektur}
\label{sec:AdvisorArchitektur}

Der WiSo@visor verfolgt ein weitgehend objektorientiertes Design. Dabei wird getrennt zwischen Framework-, Model- und Use-Case-Klassen (siehe Abbildung \ref{fig:adv-arch}).

\begin{figure}[h]
	\centering
		\includegraphics[width=0.5\textwidth]{chapter03/adv-arch.png}
	\caption{Architektur des WiSo@visor}
	\label{fig:adv-arch}
	\cite<eigene Darstellung, in Anlehnung an>{Strecker:2006}
\end{figure}

Das \textbf{Framework} des WiSo@visor �bernimmt dabei die grundlegenden Aufgaben der Applikation. Darunter fallen die zentrale Konfiguration, die Datenbankanbindung, die Benutzer- und Sitzungsverwaltung sowie Generatoren zum Erzeugen von HTML-Code �ber Templates. Daneben stellt das Framework einen zentralen Einstiegspunkt in Form der Datei "`index.php"' bereit, die je nach �bergebenem Parameter den eigentlichen Use-Case instanziiert. 

Die \textbf{Model-Klassen} abstrahieren die Datenbankzugriffe. F�r jede Tabelle in der Datenbank existiert eine Model-Klasse, die die �blichen SQL-Methoden\abbrev{SQL}{Structured Query Language} (Select, Update, Insert, Delete) bereitstellt. �ber Getter- und Setter-Methoden kann auf einzelne Attribute zugegriffen werden.

Die \textbf{Use-Case-Klassen} bilden die eigentliche Anwendungslogik. Use-Cases lassen sich f�r verschiedene Ausgabearten definieren, so etwa f�r die Ausgabe von Grafikdateien oder PDF-Dokumente. Die am h�ufigsten vorkommende Form von Use-Cases sind allerdings HTML-Ausgaben. 

~

Der gesamte Ablauf der dem WiSo@visor zu Grunde liegenden Logik gestaltet sich wie folgt:

\begin{enumerate}
	\item Der Benutzer fordert eine Seite an, z. B. �ber den Men�punkt "`Deine Nutzerdaten"' (exemplarische URL\abbrev{URL}{Unified Resource Locator}: http://server.de/wisoadvisor/index.php?action=changeuserdata).
	\item Das Framework wertet den Parameter "`action"' aus und instanziiert den zum Wert dieses Parameters geh�rigen Use-Case (hier: "`ucChangeUserData"'). Die Zuordnung erfolgt dabei im Mapping-Abschnitt des zentralen Konfigurationsobjektes (Datei "`/configuration.php"').
	\item Das Framework ruft die Methode "`execute"' des Use-Case-Objektes auf.
	\item Das Use-Case-Objekt pr�ft den evtl.\abbrev{evtl.}{eventuell} �bergebenen Parameter "`step"', um den erforderlichen Handlungsschritt zu ermitteln, und ruft dann die zu diesem Handlungsschritt geh�rige Methode auf.
	\item Diese Methode definiert die Art der Ausgabe; im Normalfall handelt es sich um HTML-Ausgaben.
	\item Bei einer HTML-Ausgabe bef�llt die Methode nun mit Hilfe der Generator-Klassen des Frameworks eine Template-Datei; die Werte daf�r stammen im Normalfall aus der Datenbank und werden �ber die Model-Klassen geladen. Zudem k�nnen hier auch schreibende Aktivit�ten erfolgen, z. B. das Speichern eines ver�nderten Datensatzes.
	\item Die Beantwortung der Anfrage ist damit abgeschlossen. Der Benutzer f�hrt nun Aktivit�ten durch (hier bspw: �nderung seiner E-Mail-Adresse) und stellt dann eine neue Anfrage (hier: Speichern der ge�nderten Daten). 
\end{enumerate}