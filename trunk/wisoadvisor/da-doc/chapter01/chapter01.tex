\section{Einleitung}
\label{sec:Einleitung}

\subsection{Motivation}

[...]

\subsection{Zielsetzung}

[...]

\subsection{Aufbau der Arbeit}

Zun�chst werden in Kapitel \ref{sec:BildungsevaluationUndBildungscontrolling} Grundlagen zu Bildungscontrolling und Bildungsevaluation erl�utert. Diese Grundlagen werden dann auf die studentische Situation an der WiSo transferiert, was in einem mehrstufigen Konzept f�r ein Controlling-Instrument f�r Studenten resultiert.

Die erste S�ule dieses Konzeptes, die bereits realisierte Studieneingungsberatung, wird in Kapitel \ref{sec:GrundlegendesAdvisor} aus p�dagogischer und technischer Sicht zusammengefasst. 

Die Kapitel \ref{sec:Studieneingangsberatung}, \ref{sec:Studienverlaufsplanung} und \ref{sec:PerformanceOptimizer} widmen sich den im Rahmen dieser Arbeit implementierten S�ulen des Konzepts: der Studieneingangsberatung f�r Studienanf�nger, der Studienverlaufsplanung sowie dem sog. "`Performance Optimizer"'.

Weitere Bestandteile des Controlling-Konzeptes werden in Kapitel \ref{sec:WeiteresAusblick} als Ausblick aufgegriffen.