\section{Import von Notendaten}
\label{sec:ImportNotendaten}

Die in der Software verwendeten statistischen Daten (Durchschnittswerte, H�ufigkeitsverteilungen) werden allesamt von der Model-Klasse "`ScheduleEntryStatistics"' ermittelt. Derzeit greift diese Klasse nur auf die von den Benutzern eingegebenen Daten zu.

Sollte eines Tages ein Import der tats�chlichen Daten des Pr�fungsamtes m�glich werden, l�sst sich �ber den globalen Parameter "`ucImporter"' => "`useimportedmarks"' die Datenbasis dieser Klasse umstellen: Ist der Wert des Parameters auf "`false"' gesetzt, werden wie gehabt die Benutzereingaben aus der Tabelle "`advisor\_\_schedule"' verwendet. 

Setzt man den Wert dagegen auf "`true"', kann zu den importierten Daten umgestellt werden. Es wird n�tig sein, neue Datenbanktabellen f�r die importierten Daten zu erstellen. Die SQL-Routinen der Klasse "`ScheduleEntryStatistics"' m�ssen auf diese neuen Tabellen abgestimmt werden (Datei "`/configuration.php"', Funktion "`configureSql"').

Um die Daten hochzuladen, ist ein Use-Case "`ucImporter"' angelegt, der von den Administratoren des Systems im Men� ausw�hlbar ist. Der detaillierte Import der Daten konnte aufgrund der unklaren Datenstrukturen jedoch nicht implementiert werden. 


