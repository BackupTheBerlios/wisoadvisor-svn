\section{Softwareentwicklung}

\subsection{Verwendete CASE-Tools}
\abbrev{CASE}{Computer Aided Software Engineering}

Die folgende Tabelle gibt eine �bersicht der im Rahmen dieser Arbeit verwendeten CASE-Tools und ihrer Einsatzzwecke.

\begin{table}[ht]
\small
	\centering
		\begin{tabular}{|l|l|l|}

			\hline
			\textbf{Programm}      & \textbf{Zweck} \\
			\hline
			Eclipse 3.2.2          & Entwicklungsumgebung \\
			\hline
			PDT                    & Plugin f�r Eclipse zur PHP-Entwicklung \\
			\hline
			Subclipse              & Plugin f�r Eclipse zur Versionsverwaltung mit Subversion \\
			\hline
			XAMPP 1.5.3a           & Lokaler Apache-Webserver mit mySQL, PHP und phpMyAdmin \\
			\hline
			DBDesigner 4           & Datenbankdesign und -pflege \\
			\hline
			BerliOS                & Host f�r Projektmanagement und Versionsverwaltung \\
			\hline
		\end{tabular}
	\caption{Verwendete CASE-Tools}
	\label{tab:casetools}
\end{table}


\subsection{Hinzuf�gen neuer Komponenten zum WiSo@visor}

Die folgende Kurzanleitung zeigt, wie ein neuer Use-Case namens "`ucHello"' samt HTML-Ausgabe �ber ein Template zum WiSo@visor hinzugef�gt wird. Damit soll es k�nftigen Entwicklern erleichtert werden, den WiSo@visor um neue Komponenten zu erweitern.

\begin{enumerate}
	\item Verzeichnis \textbf{/classes}: Neue Datei "`uc\_hello.php"' mit der von der Klasse "`UseCase"' abgeleiteten Klasse "`ucHello"'anlegen.
	\item Klasse \textbf{ucHello}: Die Methode "`execute"' (public) anlegen.
	\item Klasse \textbf{ucHello}: Eine Methode "`sayHello"' anlegen; diese Methode aus der Methode "`execute"' aufrufen.
	\item Datei \textbf{/configuration.php}: Klasse "`ucHello"' unter den "`anwendungseigenen Klassen"' bekannt machen: \\
		 	\$this->setConfValue('class', 'ucHello', null, \$classPath.'uc\_hello'.\$phpClassSuffix);
  \item Datei \textbf{/configuration.php}: Klasse "`ucHello"' zum Wert des Action-Parameters zuordnen ("`Matching"'): \\
  	 	\$this->setConfValue('usecase', 'hello', null, 'ucHello');
\end{enumerate}

Damit kann der Use-Case �ber die URL \texttt{http://server/index.php?action=hello} aufgerufen werden.

~

Die folgenden Schritte zeigen nun, wie die oben angelegte Methode "`sayHello"' �ber die Template-Engine einen Text ausgibt.

\begin{enumerate}
	\item Verzeichnis \textbf{/templates}: Neues Unterverzeichnis "`ucHello"' anlegen.
	\item Verzeichnis \textbf{/templates/ucHello}: Neue Datei "`hello.tpl"' anlegen.
	\item Datei \textbf{hello.tpl}: Exemplarisch folgenden HTML-Code verwenden: \\
	<p>Hello \#\#\#:\#\#\#replacement\#\#\#:\#\#\#!</p>
	\item Datei \textbf{/configuration.php}: Methode "`configureUcHello"' (private) anlegen und diese am Ende des Konstruktors aufrufen.
  \item Datei \textbf{/configuration.php, configureUcHello}: Die Template-Datei bekannt machen: \\
  		\$this->setConfValue('ucHello', 'htmltemplate', null, 'templates/ucHello/hello.tpl');
  \item Datei \textbf{/configuration.php, configureUcHello}: Das im Template zu ersetzende Token bekannt machen:
	    \$this->setConfValue('ucHello', 'replacement', null, 'replacement');		
  \item Klasse \textbf{ucHello, sayHello}: Ein Objekt der Klasse "`HtmlGenerator"' instanziieren: \\
      \$gen = new HtmlGenerator(\$this->getConf()->getConfString('ucHello', 'htmltemplate'), \\
                                \$this->getConf()->getConfString('template', 'indicator', 'pre'), \\
                                \$this->getConf()->getConfString('template', 'indicator', 'after'));
   \item Klasse \textbf{ucHello, sayHello}: Dem Generator-Objekt mitteilen, welches Token durch welchen Wert ersetzt werden soll: \\
       \$name = 'Qwert Zuiop�'\\
       \$gen->apply(\$this->getConf()->getConfString('ucHello', 'replacement'), \$name);
   \item Klasse \textbf{ucHello, sayHello}: Dem Generator-Objekt den HTML-Code generieren lassen und an die Ausgabe des Use-Cases anh�ngen: \\
     \$this->appendOutput(\$gen->getHTML()); \\
     \$this->setOutputType(USECASE\_HTML)); 
\end{enumerate}

Der Use-Case gibt nun bei Aufruf den Satz "`Hello Qwert Zuiop�!"' wieder. Sinnvoller als eine statischer einzusetzender Wert ist nat�rlich ein dynamisch aus der Datenbank geladener. Mit den Anweisungen dieser Kurzanleitung d�rften allerdings die ersten gro�en H�rden �berwunden werden k�nnen.